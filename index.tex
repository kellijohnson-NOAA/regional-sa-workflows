% Options for packages loaded elsewhere
\PassOptionsToPackage{unicode}{hyperref}
\PassOptionsToPackage{hyphens}{url}
\PassOptionsToPackage{dvipsnames,svgnames,x11names}{xcolor}
%
\documentclass[
  letterpaper,
  DIV=11,
  numbers=noendperiod]{scrreprt}

\usepackage{amsmath,amssymb}
\usepackage{iftex}
\ifPDFTeX
  \usepackage[T1]{fontenc}
  \usepackage[utf8]{inputenc}
  \usepackage{textcomp} % provide euro and other symbols
\else % if luatex or xetex
  \usepackage{unicode-math}
  \defaultfontfeatures{Scale=MatchLowercase}
  \defaultfontfeatures[\rmfamily]{Ligatures=TeX,Scale=1}
\fi
\usepackage{lmodern}
\ifPDFTeX\else  
    % xetex/luatex font selection
\fi
% Use upquote if available, for straight quotes in verbatim environments
\IfFileExists{upquote.sty}{\usepackage{upquote}}{}
\IfFileExists{microtype.sty}{% use microtype if available
  \usepackage[]{microtype}
  \UseMicrotypeSet[protrusion]{basicmath} % disable protrusion for tt fonts
}{}
\makeatletter
\@ifundefined{KOMAClassName}{% if non-KOMA class
  \IfFileExists{parskip.sty}{%
    \usepackage{parskip}
  }{% else
    \setlength{\parindent}{0pt}
    \setlength{\parskip}{6pt plus 2pt minus 1pt}}
}{% if KOMA class
  \KOMAoptions{parskip=half}}
\makeatother
\usepackage{xcolor}
\setlength{\emergencystretch}{3em} % prevent overfull lines
\setcounter{secnumdepth}{5}
% Make \paragraph and \subparagraph free-standing
\ifx\paragraph\undefined\else
  \let\oldparagraph\paragraph
  \renewcommand{\paragraph}[1]{\oldparagraph{#1}\mbox{}}
\fi
\ifx\subparagraph\undefined\else
  \let\oldsubparagraph\subparagraph
  \renewcommand{\subparagraph}[1]{\oldsubparagraph{#1}\mbox{}}
\fi


\providecommand{\tightlist}{%
  \setlength{\itemsep}{0pt}\setlength{\parskip}{0pt}}\usepackage{longtable,booktabs,array}
\usepackage{calc} % for calculating minipage widths
% Correct order of tables after \paragraph or \subparagraph
\usepackage{etoolbox}
\makeatletter
\patchcmd\longtable{\par}{\if@noskipsec\mbox{}\fi\par}{}{}
\makeatother
% Allow footnotes in longtable head/foot
\IfFileExists{footnotehyper.sty}{\usepackage{footnotehyper}}{\usepackage{footnote}}
\makesavenoteenv{longtable}
\usepackage{graphicx}
\makeatletter
\def\maxwidth{\ifdim\Gin@nat@width>\linewidth\linewidth\else\Gin@nat@width\fi}
\def\maxheight{\ifdim\Gin@nat@height>\textheight\textheight\else\Gin@nat@height\fi}
\makeatother
% Scale images if necessary, so that they will not overflow the page
% margins by default, and it is still possible to overwrite the defaults
% using explicit options in \includegraphics[width, height, ...]{}
\setkeys{Gin}{width=\maxwidth,height=\maxheight,keepaspectratio}
% Set default figure placement to htbp
\makeatletter
\def\fps@figure{htbp}
\makeatother
% definitions for citeproc citations
\NewDocumentCommand\citeproctext{}{}
\NewDocumentCommand\citeproc{mm}{%
  \begingroup\def\citeproctext{#2}\cite{#1}\endgroup}
\makeatletter
 % allow citations to break across lines
 \let\@cite@ofmt\@firstofone
 % avoid brackets around text for \cite:
 \def\@biblabel#1{}
 \def\@cite#1#2{{#1\if@tempswa , #2\fi}}
\makeatother
\newlength{\cslhangindent}
\setlength{\cslhangindent}{1.5em}
\newlength{\csllabelwidth}
\setlength{\csllabelwidth}{3em}
\newenvironment{CSLReferences}[2] % #1 hanging-indent, #2 entry-spacing
 {\begin{list}{}{%
  \setlength{\itemindent}{0pt}
  \setlength{\leftmargin}{0pt}
  \setlength{\parsep}{0pt}
  % turn on hanging indent if param 1 is 1
  \ifodd #1
   \setlength{\leftmargin}{\cslhangindent}
   \setlength{\itemindent}{-1\cslhangindent}
  \fi
  % set entry spacing
  \setlength{\itemsep}{#2\baselineskip}}}
 {\end{list}}
\usepackage{calc}
\newcommand{\CSLBlock}[1]{\hfill\break\parbox[t]{\linewidth}{\strut\ignorespaces#1\strut}}
\newcommand{\CSLLeftMargin}[1]{\parbox[t]{\csllabelwidth}{\strut#1\strut}}
\newcommand{\CSLRightInline}[1]{\parbox[t]{\linewidth - \csllabelwidth}{\strut#1\strut}}
\newcommand{\CSLIndent}[1]{\hspace{\cslhangindent}#1}

\KOMAoption{captions}{tableheading}
\usepackage{nameref}
\makeatletter
\@ifpackageloaded{bookmark}{}{\usepackage{bookmark}}
\makeatother
\makeatletter
\@ifpackageloaded{caption}{}{\usepackage{caption}}
\AtBeginDocument{%
\ifdefined\contentsname
  \renewcommand*\contentsname{Table of contents}
\else
  \newcommand\contentsname{Table of contents}
\fi
\ifdefined\listfigurename
  \renewcommand*\listfigurename{List of Figures}
\else
  \newcommand\listfigurename{List of Figures}
\fi
\ifdefined\listtablename
  \renewcommand*\listtablename{List of Tables}
\else
  \newcommand\listtablename{List of Tables}
\fi
\ifdefined\figurename
  \renewcommand*\figurename{Figure}
\else
  \newcommand\figurename{Figure}
\fi
\ifdefined\tablename
  \renewcommand*\tablename{Table}
\else
  \newcommand\tablename{Table}
\fi
}
\@ifpackageloaded{float}{}{\usepackage{float}}
\floatstyle{ruled}
\@ifundefined{c@chapter}{\newfloat{codelisting}{h}{lop}}{\newfloat{codelisting}{h}{lop}[chapter]}
\floatname{codelisting}{Listing}
\newcommand*\listoflistings{\listof{codelisting}{List of Listings}}
\makeatother
\makeatletter
\makeatother
\makeatletter
\@ifpackageloaded{caption}{}{\usepackage{caption}}
\@ifpackageloaded{subcaption}{}{\usepackage{subcaption}}
\makeatother
\ifLuaTeX
  \usepackage{selnolig}  % disable illegal ligatures
\fi
\usepackage{bookmark}

\IfFileExists{xurl.sty}{\usepackage{xurl}}{} % add URL line breaks if available
\urlstyle{same} % disable monospaced font for URLs
\hypersetup{
  pdftitle={Summary of U.S. Stock Assessment Workflows},
  pdfauthor={Samantha Schiano},
  colorlinks=true,
  linkcolor={blue},
  filecolor={Maroon},
  citecolor={Blue},
  urlcolor={Blue},
  pdfcreator={LaTeX via pandoc}}

\title{Summary of U.S. Stock Assessment Workflows}
\usepackage{etoolbox}
\makeatletter
\providecommand{\subtitle}[1]{% add subtitle to \maketitle
  \apptocmd{\@title}{\par {\large #1 \par}}{}{}
}
\makeatother
\subtitle{Tools, Templates, and other Resources}
\author{Samantha Schiano}
\date{2024-04-01}

\begin{document}
\maketitle

\renewcommand*\contentsname{Table of contents}
{
\hypersetup{linkcolor=}
\setcounter{tocdepth}{2}
\tableofcontents
}
\bookmarksetup{startatroot}

\chapter*{Citation}\label{citation}
\addcontentsline{toc}{chapter}{Citation}

\markboth{Citation}{Citation}

Schiano, S. 2024. Summary of U.S. Stock Assessment Workflows: Tools and
Templates. NOAA Fisheries Office of Science and Technology.

\bookmarksetup{startatroot}

\chapter{Summary}\label{summary}

Developing and producing a stock assessment report requires a
considerable amount of data consolidation, analysis, and research.
Reports can range from 10 pages to well over 300, but the goal of the
process remains to provide management advice with the current and
projected status of the stock, catches, and other important parameters
to ensure its sustainability. There have been many efforts done to
improve the reproducibility of workflows and reduce time it takes to
produce these reports. So far,improvements and efforts have ranged
considerably by region of the U.S. Specifically, each of the seven
regional fishery science centers across the U.S. have their own
workflows to assess a stock and produce its report. Many of these
workflows are guided by requirements from fishery management councils
and other involved managing bodies that utilize these reports to
delegate fishery regulations.

Workflows across the country not only range from region to region, but
from scientist to scientist. There is no current standardized or
accepted best practice for producing a stock assessment report. Some
centers rely on the use of latex, a common software to produce documents
with in-text calculations, while others utilize more recently developed
programs like Rmarkdown and quarto, which both use latex as a basis for
their production. Others utilize the capacity of Microsoft word which
restricts the reproducibility of a report as well as increases work time
since figures, tables, and other associated resources must be compiled
outside of the word document.

\textbf{Description of interaction and list of fishery management
coucils across the US}

\bookmarksetup{startatroot}

\chapter{General Stock Assessment Workflow (commonalities and
process)}\label{sec-general}

\begin{itemize}
\tightlist
\item
  Description of stock assessment workflows from input to bringing the
  report to the SSC and council(s) for evaluation and adoption of formal
  management measures resulting from recommendations
\end{itemize}

\begin{enumerate}
\def\labelenumi{\arabic{enumi}.}
\item
  Gather data inputs for model
\item
  Configure assessment model with updated data (based on assessment need
  for that year)
\item
  Sensitivity runs and projections
\item
  Develop assessment report for SSC, councils, and/or RFMOs
\item
  Present assessment and recommendations to SSC and councils
\item
  Assessment accepted or not, create formal report for public release
  and adoption recommendations as designated by the councils
\end{enumerate}

\bookmarksetup{startatroot}

\chapter{Stock Assessment Models}\label{stock-assessment-models}

{[}Short descriptions of commonly used stock assessment models within
the U.S. including acknowledgement of smaller used models and FIMS for
the future{]}

\begin{itemize}
\item
  General use of assessment models in the workflow (self-explanatory)
\item
  Short descriptions of various models used in the U.S. around the
  regions (purpose is so that the reader can understand the model when
  it is reference in the section later)

  \begin{itemize}
  \tightlist
  \item
    Categorize based on assessment type rather than assessment model
    (age-structured/catch-at-age, catch-at-length, VPA, Agg. Biomass
    Dynamics, Index-based, data-limited)
  \end{itemize}
\item
  Include link/reference to papers/repositories at end of summary for
  reader to reference (also refer to FIT)
\item
  WHAM, SS, BAM, ASAP, AMAK, Bespoke, FIMS, ect
\end{itemize}

\bookmarksetup{startatroot}

\chapter{Tools and Resources}\label{sec-tools}

\begin{itemize}
\tightlist
\item
  Tools and templates available for different models and workflows
  (added as a list of tools in a way instead of by region since there
  are a lot of regions that use the same tools)
\end{itemize}

\section{afscdata}\label{sec-afscdata}

\section{afscassess}\label{sec-afscassess}

\section{safe Report Template}\label{sec-safe}

\section{r4ss}\label{sec-r4ss}

\section{sa4ss - SS Report Template}\label{sec-sa4ss}

\section{ASAPplots}\label{sec-asapplots}

\section{MAFSC SAFE Reports Template}\label{sec-mafscsafe}

\section{SEDAR-Assessement Report Template}\label{sec-sedartemplate}

\section{FishGraph}\label{sec-fishgraph}

\section{ADMB2R}\label{sec-admb2r}

\section{SASINF}\label{sec-sasinf}

\section{SW Stock Assessment Template}\label{sec-swfsctemplate}

\section{BrailleR}\label{sec-brailler}

\section{SW R Process Output}\label{sec-swroutput}

\section{swfscMisc}\label{sec-swfscmisc}

\section{NMFSReports}\label{sec-nmfsreports}

\section{NOAA Tech Memo Template}\label{sec-noaatechmemo}

\section{Other Assessment Report Template
Repositories}\label{sec-othertemplates}

\bookmarksetup{startatroot}

\chapter{Region Specific Workflow}\label{sec-regionworkflows}

\begin{itemize}
\item
  Descriptions of the workflows by region and what separates them from
  other regions
\item
  Advances made by this region and the tools they use
\item
  Particular struggles or unique operations incorporated into their
  workflow
\item
  Table of resources used for their workflows
\end{itemize}

\section{AFSC}\label{sec-afsc}

Alternative way to reference section \nameref{sec-safe} rather than
Section~\ref{sec-safe}

\section{NEFSC}\label{sec-nefsc}

Process variations:

\begin{itemize}
\item
  Informal/verbal agreement for TOR guidelines for each stock
\item
  Standardized report template agreed upon by NOAA and the MAFMC AND
  NEFMC (agreed on in 2017)

  \begin{itemize}
  \tightlist
  \item
    Short and concise to make policy decisions (mgmt track specifically)
  \end{itemize}
\item
  Report template is not publicly available but all done in latex
  (modular workflow)

  \begin{itemize}
  \item
    Figures rendered outside (saved as png) then reference in doc
  \item
    Tables created into tex files and referenced as component in
    template
  \end{itemize}
\item
  Extensive work with 508 compliance

  \begin{itemize}
  \tightlist
  \item
    Contractors developing package for compliance to apply to template
  \end{itemize}
\end{itemize}

Largest problems:

\begin{itemize}
\item
  Lots of processing variables for 508 compliance
\item
  Even with all the work in the compliance, there is still a large
  effort into making it accessible (\textasciitilde2 week conversion for
  single analyst at the center)
\item
\end{itemize}

\section{NWFSC}\label{sec-nwfsc}

\section{PIFSC}\label{sec-pifsc}

\section{SEFSC}\label{sec-sefsc}

\section{SWFSC}\label{sec-swfsc}

\bookmarksetup{startatroot}

\chapter{Conclusions}\label{conclusions}

\begin{itemize}
\item
  Conclusions about how workflows operate in the U.S.

  \begin{itemize}
  \item
    Both pros and pitfalls of current workflows
  \item
    Incorporating perspectives from regional scientists?
  \end{itemize}
\item
  Additional hope for greater look into the inputs and pulling together
  the entire stock assessment workflow process
\end{itemize}

\section{Future Work}\label{future-work}

\begin{itemize}
\tightlist
\item
  Discuss the upcoming/in development tool for automated workflows and
  plan for the future including stock assessment modelling for the
  future
\end{itemize}

\bookmarksetup{startatroot}

\chapter*{References}\label{references}
\addcontentsline{toc}{chapter}{References}

\markboth{References}{References}

\phantomsection\label{refs}
\begin{CSLReferences}{0}{1}
\end{CSLReferences}



\end{document}
