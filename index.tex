% Options for packages loaded elsewhere
\PassOptionsToPackage{unicode}{hyperref}
\PassOptionsToPackage{hyphens}{url}
%
\documentclass[
  letterpaper,
  oneside,
  open=any]{scrbook}

\usepackage{amsmath,amssymb}
\usepackage{iftex}
\ifPDFTeX
  \usepackage[T1]{fontenc}
  \usepackage[utf8]{inputenc}
  \usepackage{textcomp} % provide euro and other symbols
\else % if luatex or xetex
  \usepackage{unicode-math}
  \defaultfontfeatures{Scale=MatchLowercase}
  \defaultfontfeatures[\rmfamily]{Ligatures=TeX,Scale=1}
\fi
\usepackage{lmodern}
\ifPDFTeX\else  
    % xetex/luatex font selection
\fi
% Use upquote if available, for straight quotes in verbatim environments
\IfFileExists{upquote.sty}{\usepackage{upquote}}{}
\IfFileExists{microtype.sty}{% use microtype if available
  \usepackage[]{microtype}
  \UseMicrotypeSet[protrusion]{basicmath} % disable protrusion for tt fonts
}{}
\makeatletter
\@ifundefined{KOMAClassName}{% if non-KOMA class
  \IfFileExists{parskip.sty}{%
    \usepackage{parskip}
  }{% else
    \setlength{\parindent}{0pt}
    \setlength{\parskip}{6pt plus 2pt minus 1pt}}
}{% if KOMA class
  \KOMAoptions{parskip=half}}
\makeatother
\usepackage{xcolor}
\setlength{\emergencystretch}{3em} % prevent overfull lines
\setcounter{secnumdepth}{5}
% Make \paragraph and \subparagraph free-standing
\ifx\paragraph\undefined\else
  \let\oldparagraph\paragraph
  \renewcommand{\paragraph}[1]{\oldparagraph{#1}\mbox{}}
\fi
\ifx\subparagraph\undefined\else
  \let\oldsubparagraph\subparagraph
  \renewcommand{\subparagraph}[1]{\oldsubparagraph{#1}\mbox{}}
\fi

\providecommand{\tightlist}{%
  \setlength{\itemsep}{0pt}\setlength{\parskip}{0pt}}\usepackage{longtable,booktabs,array}
\usepackage{calc} % for calculating minipage widths
% Correct order of tables after \paragraph or \subparagraph
\usepackage{etoolbox}
\makeatletter
\patchcmd\longtable{\par}{\if@noskipsec\mbox{}\fi\par}{}{}
\makeatother
% Allow footnotes in longtable head/foot
\IfFileExists{footnotehyper.sty}{\usepackage{footnotehyper}}{\usepackage{footnote}}
\makesavenoteenv{longtable}
\usepackage{graphicx}
\makeatletter
\def\maxwidth{\ifdim\Gin@nat@width>\linewidth\linewidth\else\Gin@nat@width\fi}
\def\maxheight{\ifdim\Gin@nat@height>\textheight\textheight\else\Gin@nat@height\fi}
\makeatother
% Scale images if necessary, so that they will not overflow the page
% margins by default, and it is still possible to overwrite the defaults
% using explicit options in \includegraphics[width, height, ...]{}
\setkeys{Gin}{width=\maxwidth,height=\maxheight,keepaspectratio}
% Set default figure placement to htbp
\makeatletter
\def\fps@figure{htbp}
\makeatother
% definitions for citeproc citations
\NewDocumentCommand\citeproctext{}{}
\NewDocumentCommand\citeproc{mm}{%
  \begingroup\def\citeproctext{#2}\cite{#1}\endgroup}
\makeatletter
 % allow citations to break across lines
 \let\@cite@ofmt\@firstofone
 % avoid brackets around text for \cite:
 \def\@biblabel#1{}
 \def\@cite#1#2{{#1\if@tempswa , #2\fi}}
\makeatother
\newlength{\cslhangindent}
\setlength{\cslhangindent}{1.5em}
\newlength{\csllabelwidth}
\setlength{\csllabelwidth}{3em}
\newenvironment{CSLReferences}[2] % #1 hanging-indent, #2 entry-spacing
 {\begin{list}{}{%
  \setlength{\itemindent}{0pt}
  \setlength{\leftmargin}{0pt}
  \setlength{\parsep}{0pt}
  % turn on hanging indent if param 1 is 1
  \ifodd #1
   \setlength{\leftmargin}{\cslhangindent}
   \setlength{\itemindent}{-1\cslhangindent}
  \fi
  % set entry spacing
  \setlength{\itemsep}{#2\baselineskip}}}
 {\end{list}}
\usepackage{calc}
\newcommand{\CSLBlock}[1]{\hfill\break\parbox[t]{\linewidth}{\strut\ignorespaces#1\strut}}
\newcommand{\CSLLeftMargin}[1]{\parbox[t]{\csllabelwidth}{\strut#1\strut}}
\newcommand{\CSLRightInline}[1]{\parbox[t]{\linewidth - \csllabelwidth}{\strut#1\strut}}
\newcommand{\CSLIndent}[1]{\hspace{\cslhangindent}#1}

\usepackage[default]{opensans}
\fontseries{lc}\selectfont
\definecolor{nmfsblue1}{HTML}{00467f}
\definecolor{nmfsblue2}{HTML}{007eb2}
\makeatletter
\@ifpackageloaded{bookmark}{}{\usepackage{bookmark}}
\makeatother
\makeatletter
\@ifpackageloaded{caption}{}{\usepackage{caption}}
\AtBeginDocument{%
\ifdefined\contentsname
  \renewcommand*\contentsname{Table of contents}
\else
  \newcommand\contentsname{Table of contents}
\fi
\ifdefined\listfigurename
  \renewcommand*\listfigurename{List of Figures}
\else
  \newcommand\listfigurename{List of Figures}
\fi
\ifdefined\listtablename
  \renewcommand*\listtablename{List of Tables}
\else
  \newcommand\listtablename{List of Tables}
\fi
\ifdefined\figurename
  \renewcommand*\figurename{Figure}
\else
  \newcommand\figurename{Figure}
\fi
\ifdefined\tablename
  \renewcommand*\tablename{Table}
\else
  \newcommand\tablename{Table}
\fi
}
\@ifpackageloaded{float}{}{\usepackage{float}}
\floatstyle{ruled}
\@ifundefined{c@chapter}{\newfloat{codelisting}{h}{lop}}{\newfloat{codelisting}{h}{lop}[chapter]}
\floatname{codelisting}{Listing}
\newcommand*\listoflistings{\listof{codelisting}{List of Listings}}
\makeatother
\makeatletter
\makeatother
\makeatletter
\@ifpackageloaded{caption}{}{\usepackage{caption}}
\@ifpackageloaded{subcaption}{}{\usepackage{subcaption}}
\makeatother

\usepackage{hyphenat}
\usepackage{ifthen}
\usepackage{calc}
\usepackage{calculator}

\usepackage{graphicx}
\usepackage{wallpaper}

\usepackage{geometry}

\usepackage{graphicx}
\usepackage{geometry}
\usepackage{afterpage}
\usepackage{tikz}
\usetikzlibrary{calc}
\usetikzlibrary{fadings}
\usepackage[pagecolor=none]{pagecolor}


% Set the titlepage font families







% Set the coverpage font families

\ifLuaTeX
  \usepackage{selnolig}  % disable illegal ligatures
\fi
\usepackage{bookmark}

\IfFileExists{xurl.sty}{\usepackage{xurl}}{} % add URL line breaks if available
\urlstyle{same} % disable monospaced font for URLs
\hypersetup{
  pdftitle={Summary of U.S. Stock Assessment Workflows},
  pdfauthor={Samantha Schiano},
  hidelinks,
  pdfcreator={LaTeX via pandoc}}

\title{Summary of U.S. Stock Assessment Workflows}
\usepackage{etoolbox}
\makeatletter
\providecommand{\subtitle}[1]{% add subtitle to \maketitle
  \apptocmd{\@title}{\par {\large #1 \par}}{}{}
}
\makeatother
\subtitle{Tools and Templates}
\author{Samantha Schiano}
\date{2024-04-01}

\begin{document}
%%%%% begin titlepage extension code

  \begin{frontmatter}

\begin{titlepage}
% This is a combination of Pandoc templating and LaTeX
% Pandoc templating https://pandoc.org/MANUAL.html#templates
% See the README for help

\thispagestyle{empty}

\newgeometry{top=-100in}

% Page color

\newcommand{\coverauthorstyle}[1]{{\fontsize{20}{24.0}\selectfont
#1}}

\begin{tikzpicture}[remember picture, overlay, inner sep=0pt, outer sep=0pt]

\tikzfading[name=fadeout, inner color=transparent!0,outer color=transparent!100]
\tikzfading[name=fadein, inner color=transparent!100,outer color=transparent!0]
\node[anchor=south west, rotate=0.0, opacity=1.0] at ($(current page.south west)+(0pt, 8.75in)$) {
\includegraphics[width=\paperwidth, keepaspectratio]{images/cover-header-2.png}};

% Title
\newcommand{\titlelocationleft}{2.3in}
\newcommand{\titlelocationbottom}{7in}
\newcommand{\titlealign}{left}

\begin{scope}{%
\fontsize{50}{60.0}\selectfont
\node[anchor=north
west, align=left, rotate=0] (Title1) at ($(current page.south west)+(\titlelocationleft,\titlelocationbottom)$)  [text width = 5in]  {\textcolor{nmfsblue1}{\nohyphens{Summary
of U.S. Stock Assessment Workflows}}};
}
\end{scope}

% Author
\newcommand{\authorlocationleft}{2.3in}
\newcommand{\authorlocationbottom}{5in}
\newcommand{\authoralign}{left}

\begin{scope}
{%
\fontsize{20}{24.0}\selectfont
\node[anchor=north
west, align=left, rotate=0] (Author1) at ($(current page.south west)+(\authorlocationleft,\authorlocationbottom)$)  [text width = 5in]  {
\coverauthorstyle{Samantha Schiano\\}};
}
\end{scope}

% Header
\newcommand{\headerlocationleft}{2.3in}
\newcommand{\headerlocationbottom}{9.8in}
\newcommand{\headerlocationalign}{left}

\begin{scope}
{%
\fontsize{16}{19.2}\selectfont
 \node[anchor=north west, align=left, rotate=0] (Header1) at %
($(current page.south west)+(\headerlocationleft,\headerlocationbottom)$)  [text width = 5in]  {\textcolor{white}{\nohyphens{NOAA
Technical Memorandum NMFS-XXX-\#\#}}};
}
\end{scope}

% Footer
\newcommand{\footerlocationleft}{6in}
\newcommand{\footerlocationbottom}{0.1\paperheight}
\newcommand{\footerlocationalign}{left}

\begin{scope}
{%
\fontsize{8}{9.6}\selectfont
 \node[anchor=north west, align=left, rotate=0] (Footer1) at %
($(current page.south west)+(\footerlocationleft,\footerlocationbottom)$)  [text width = 2.5in]  {{\nohyphens{U.S.
DEPARTMENT OF COMMERCE\\
\strut \\
National Oceanic and Atmospheric Administration\\
National Marine Fisheries Service\\
Northwest Fisheries Science Center}}};
}
\end{scope}

% Date
\newcommand{\datelocationleft}{6in}
\newcommand{\datelocationbottom}{2in}
\newcommand{\datelocationalign}{left}

\begin{scope}
{%
\fontsize{20}{24.0}\selectfont
 \node[anchor=north west, align=left, rotate=0] (Date1) at %
($(current page.south west)+(\datelocationleft,\datelocationbottom)$)  [text width = 2.5in]  {{\nohyphens{January
2023}}};
}
\end{scope}

\end{tikzpicture}
\clearpage
\restoregeometry
%%% TITLE PAGE START

% Set up alignment commands
%Page
\newcommand{\titlepagepagealign}{
\ifthenelse{\equal{left}{right}}{\raggedleft}{}
\ifthenelse{\equal{left}{center}}{\centering}{}
\ifthenelse{\equal{left}{left}}{\raggedright}{}
}
%% Titles
\newcommand{\titlepagetitlealign}{
\ifthenelse{\equal{left}{right}}{\raggedleft}{}
\ifthenelse{\equal{left}{center}}{\centering}{}
\ifthenelse{\equal{left}{left}}{\raggedright}{}
\ifthenelse{\equal{left}{spread}}{\makebox[\linewidth][s]}{}
}


\newcommand{\titleandsubtitle}{
% Title and subtitle
{\fontsize{30}{36.0}\selectfont
\textcolor{nmfsblue1}{\bfseries{\nohyphens{Summary of U.S. Stock
Assessment Workflows}}}\par
}%

\vspace{\betweentitlesubtitle}
{
{\textit{\nohyphens{Tools and Templates}}}\par
}}
\newcommand{\titlepagetitleblock}{
\titleandsubtitle
}

\newcommand{\authorstyle}[1]{{\fontsize{20}{24.0}\selectfont
#1}}

\newcommand{\affiliationstyle}[1]{{#1}}

\newcommand{\titlepageauthorblock}{
{\authorstyle{\nohyphens{Samantha Schiano}{\textsuperscript{1}}}}}

\newcommand{\titlepageaffiliationblock}{
\hangindent=1em
\hangafter=1
{\affiliationstyle{
{1}.~ECS Federal in Support of,~NOAA Fisheries Office of Science and
Technology,~1315 East West HighwaySilver Spring, MD 20910


\vspace{1\baselineskip} 
}}
}
\newcommand{\headerstyled}{%
{}
}
\newcommand{\footerstyled}{%
{}
}
\newcommand{\datestyled}{%
{2024-04-01}
}


\newcommand{\titlepageheaderblock}{\headerstyled}

\newcommand{\titlepagefooterblock}{
\footerstyled
}

\newcommand{\titlepagedateblock}{
\datestyled
}

%set up blocks so user can specify order
\newcommand{\titleblock}{\newlength{\betweentitlesubtitle}
\setlength{\betweentitlesubtitle}{1pt}
{\titlepagetitlealign

{\titlepagetitleblock}
}

\vspace{4\baselineskip}
}

\newcommand{\authorblock}{{\titlepageauthorblock}

\vspace{2\baselineskip}
}

\newcommand{\affiliationblock}{{\titlepageaffiliationblock}

\vspace{2\baselineskip}
}

\newcommand{\logoblock}{}

\newcommand{\footerblock}{}

\newcommand{\dateblock}{{\titlepagedateblock}

\vspace{0pt}
}

\newcommand{\headerblock}{}
\newgeometry{top=3in,bottom=1in,right=1in,left=1.75in}
% background image
\newlength{\bgimagesize}
\setlength{\bgimagesize}{0.75\paperwidth}
\LENGTHDIVIDE{\bgimagesize}{\paperwidth}{\theRatio} % from calculator pkg
\ThisULCornerWallPaper{\theRatio}{images/corner-image.png}

\thispagestyle{empty} % no page numbers on titlepages


\newcommand{\vrulecode}{\rule{\vrulewidth}{\textheight}}
\newlength{\vrulewidth}
\setlength{\vrulewidth}{0pt}
\newlength{\B}
\setlength{\B}{\ifdim\vrulewidth > 0pt 0.05\textwidth\else 0pt\fi}
\newlength{\minipagewidth}
\ifthenelse{\equal{left}{left} \OR \equal{left}{right} }
{% True case
\setlength{\minipagewidth}{\textwidth - \vrulewidth - \B - 0.1\textwidth}
}{
\setlength{\minipagewidth}{\textwidth - 2\vrulewidth - 2\B - 0.1\textwidth}
}
\ifthenelse{\equal{left}{left} \OR \equal{left}{leftright}}
{% True case
\raggedleft % needed for the minipage to work
\vrulecode
\hspace{\B}
}{%
\raggedright % else it is right only and width is not 0
}
% [position of box][box height][inner position]{width}
% [s] means stretch out vertically; assuming there is a vfill
\begin{minipage}[b][\textheight][s]{\minipagewidth}
\titlepagepagealign
\headerblock

\titleblock

\authorblock

\affiliationblock

\vfill

\logoblock

\footerblock
\par

\end{minipage}\ifthenelse{\equal{left}{right} \OR \equal{left}{leftright} }{
\hspace{\B}
\vrulecode}{}
\clearpage
\restoregeometry
%%% TITLE PAGE END
\end{titlepage}
\setcounter{page}{1}
\end{frontmatter}

%%%%% end titlepage extension code

\renewcommand*\contentsname{Table of contents}
{
\setcounter{tocdepth}{1}
\tableofcontents
}
\listoffigures
\listoftables
\mainmatter
\bookmarksetup{startatroot}

\chapter*{Citation}\label{citation}
\addcontentsline{toc}{chapter}{Citation}

\markboth{Citation}{Citation}

Schiano, S. 2024. Summary of U.S. Stock Assessment Workflows: Tools and
Templates. NOAA Fisheries Office of Science and Technology.

\bookmarksetup{startatroot}

\chapter{Summary}\label{summary}

Developing and producing a stock assessment report requires a
considerable amount of data consolidation, analysis, and research.
Reports can range from 10 pages to well over 300, but the goal of the
process remains to provide management advice with the current and
projected status of the stock, catches, and other important parameters
to ensure its sustainability. There have been many efforts done to
improve the reproducibility of workflows and reduce time it takes to
produce these reports. So far,improvements and efforts have ranged
considerably by region of the U.S. Specifically, each of the seven
regional fishery science centers across the U.S. have their own
workflows to assess a stock and produce its report. Many of these
workflows are guided by requirements from fishery management councils
and other involved managing bodies that utilize these reports to
delegate fishery regulations.

Workflows across the country not only range from region to region, but
from scientist to scientist. There is no current standardized or
accepted best practice for producing a stock assessment report. Some
centers rely on the use of latex, a common software to produce documents
with in-text calculations, while others utilize more recently developed
programs like Rmarkdown and quarto, which both use latex as a basis for
their production. Others utilize the capacity of Microsoft word which
restricts the reproducibility of a report as well as increases work time
since figures, tables, and other associated resources must be compiled
outside of the word document.

\textbf{Description of interaction and list of fishery management
coucils across the US}

\bookmarksetup{startatroot}

\chapter{General Stock Assessment Workflow
Structure}\label{general-stock-assessment-workflow-structure}

\begin{itemize}
\tightlist
\item
  Description of stock assessment workflows from input to bringing the
  report to the SSC and council(s) for evaluation and adoption of formal
  management measures resulting from recommendations
\end{itemize}

\bookmarksetup{startatroot}

\chapter{Stock Assessment Models}\label{stock-assessment-models}

{[}Short descriptions of commonly used stock assessment models within
the U.S. including acknowledgement of smaller used models and FIMS for
the future{]}

\begin{itemize}
\item
  WHAM
\item
  SS
\item
  BAM
\item
  ASAP
\item
  AMAK
\item
  Bespoke
\item
  FIMS
\end{itemize}

\bookmarksetup{startatroot}

\chapter{Alaska Fisheries Science Center
(AFSC)}\label{alaska-fisheries-science-center-afsc}

\section{Resources}\label{resources}

AFSC takes advantage of multiple packages and templates built to help
produce stock assessment reports. They have also built their own tools
to improve and streamline their workflows. These tools include:

\begin{itemize}
\tightlist
\item
  afscdata
\item
  afscassess
\item
  safe
\item
  r4ss
\end{itemize}

\subsection{afscdata}\label{afscdata}

\href{https://github.com/afsc-assessments/afscdata/}{Github Repository}

This R package was developed in order to extract fishery data for use in
analyses and stock assessment models. Various functions query data from
a connected database. There are various functions dependent on the
target species which extract .csv files with a time stamp of the query.

\textbf{Other tools to describe:}

\begin{itemize}
\item
  afscassess
\item
  safe reporting
\end{itemize}

\section{Workflow}\label{workflow}

\begin{itemize}
\tightlist
\item
  Full assessments vs.~partial assessments vs.~model review
\end{itemize}

\begin{verbatim}
-   Add'l harvest projections
\end{verbatim}

\begin{itemize}
\item
  Operational \& stock monitoring update
\item
  Annual and bi-annual assessments (+ some every 4 yrs)
\item
  bespoke and SS models
\end{itemize}

\subsubsection{Assessment Outlines}\label{assessment-outlines}

\begin{itemize}
\tightlist
\item
  Outlines as agreed upon in the TOR from council
\end{itemize}

\bookmarksetup{startatroot}

\chapter{Northeast Fisheries Science Center
(NEFSC)}\label{northeast-fisheries-science-center-nefsc}

\begin{itemize}
\item
  Summary of their process and broken down by subregion if applicable
\item
  Discuss impact of multiple fishery management councils and other RFMOs
  for document guidance and content delegation
\end{itemize}

\section{Resources}\label{resources-1}

\begin{itemize}
\item
  table of current resources that guide their workflow?
\item
  Alternative is descriptions of different resources used in the process
  as described in the AFSC section example
\end{itemize}

\textbf{Tools to describe:}

\begin{itemize}
\tightlist
\item
  ASAPplots
\end{itemize}

\section{Workflow/TOR}\label{workflowtor}

\begin{itemize}
\item
  Discussion of the process for the workflow in the region - described
  like methods

  \begin{itemize}
  \tightlist
  \item
    Subsections for 508 compliance efforts and tools?
  \end{itemize}
\item
\end{itemize}

Adding example citation in a sentence to test using them and allowing
render (Clark 1993).

\bookmarksetup{startatroot}

\chapter{Northwest Fisheries Science Center
(NWFSC)}\label{northwest-fisheries-science-center-nwfsc}

\begin{itemize}
\item
  Summary of their process and broken down by subregion if applicable
\item
  Discuss impact of multiple fishery management councils and other RFMOs
  for document guidance and content delegation
\end{itemize}

\section{Resources}\label{resources-2}

\begin{itemize}
\tightlist
\item
  table of current resources that guide their workflow?
\end{itemize}

\textbf{Tools to describe:}

\begin{itemize}
\tightlist
\item
  r4ss
\item
  sa4ss
\item
  SASINF
\item
  Other tools used in process?
\end{itemize}

\section{Workflow/TOR(s)}\label{workflowtors}

\begin{itemize}
\item
  Discussion of the process for the workflow in the region - described
  like methods

  \begin{itemize}
  \tightlist
  \item
    Subsections for 508 compliance and tools?
  \end{itemize}
\end{itemize}

\bookmarksetup{startatroot}

\chapter{Pacific Islands Fisheries Science Center
(PIFSC)}\label{pacific-islands-fisheries-science-center-pifsc}

\begin{itemize}
\item
  Summary of their process and broken down by subregion if applicable
\item
  Discuss impact of multiple fishery management councils and other RFMOs
  for document guidance and content delegation
\end{itemize}

\section{Resources}\label{resources-3}

\begin{itemize}
\tightlist
\item
  table of current resources that guide their workflow?
\end{itemize}

\textbf{Tools to describe:}

\begin{itemize}
\item
  \href{https://github.com/PIFSCstockassessments/AmSam-Bottomfish-2023/tree/master/Scripts/03_Report\%20scripts}{bottomfish
  report script}
\item
  Stock assessment report repo
\end{itemize}

\section{Workflow/TOR}\label{workflowtor-1}

\begin{itemize}
\item
  Discussion of the process for the workflow in the region - described
  like methods

  \begin{itemize}
  \tightlist
  \item
    Subsections for 508 compliance and tools?
  \end{itemize}
\end{itemize}

\bookmarksetup{startatroot}

\chapter{Southeast Fisheries Science Center
(SEFSC)}\label{southeast-fisheries-science-center-sefsc}

\begin{itemize}
\item
  Summary of their process and broken down by subregion if applicable
\item
  Discuss impact of multiple fishery management councils and other RFMOs
  for document guidance and content delegation
\end{itemize}

\section{Resources}\label{resources-4}

\begin{itemize}
\tightlist
\item
  table of current resources that guide their workflow?
\end{itemize}

\textbf{Tools to describe:}

\begin{itemize}
\item
  r4ss*
\item
  SEDAR-Assessment-Report template (GOM)
\item
  Latex template (SA)
\item
\end{itemize}

\section{Workflow/TOR(s)}\label{workflowtors-1}

\begin{itemize}
\item
  Discussion of the process for the workflow in the region - described
  like methods

  \begin{itemize}
  \tightlist
  \item
    Subsections for 508 compliance and tools?
  \end{itemize}
\end{itemize}

\bookmarksetup{startatroot}

\chapter{Southwest Fisheries Science Center
(SWFSC)}\label{southwest-fisheries-science-center-swfsc}

\begin{itemize}
\item
  Summary of their process and broken down by subregion if applicable
\item
  Discuss impact of multiple fishery management councils and other RFMOs
  for document guidance and content delegation
\end{itemize}

\section{Resources}\label{resources-5}

\begin{itemize}
\tightlist
\item
  table of current resources that guide their workflow?
\end{itemize}

\textbf{Tools to describe:}

\begin{itemize}
\item
  \href{https://github.com/melissamonk-NOAA/StockAssessment_template}{Stock
  assessment template}
\item
  \href{https://github.com/peterkuriyama/cpsassessment/blob/main/R/process_output.R}{R
  process output}
\item
  \href{https://github.com/EricArcher/swfscMisc/tree/master}{swfscMisc}
\item
  r4ss*
\end{itemize}

\section{Workflow/TOR}\label{workflowtor-2}

\begin{itemize}
\item
  Discussion of the process for the workflow in the region - described
  like methods

  \begin{itemize}
  \tightlist
  \item
    Subsections for 508 compliance and tools?
  \end{itemize}
\end{itemize}

\bookmarksetup{startatroot}

\chapter{Conclusions}\label{conclusions}

\begin{itemize}
\item
  Conclusions about how workflows operate in the U.S.

  \begin{itemize}
  \item
    Both pros and pitfalls of current workflows
  \item
    Incorporating perspectives from regional scientists?
  \end{itemize}
\item
  Additional hope for greater look into the inputs and pulling together
  the entire stock assessment workflow process
\end{itemize}

\section{Future Work}\label{future-work}

\begin{itemize}
\tightlist
\item
  Discuss the upcoming/in development tool for automated workflows and
  plan for the future including stock assessment modelling for the
  future
\end{itemize}

\bookmarksetup{startatroot}

\chapter*{References}\label{references}
\addcontentsline{toc}{chapter}{References}

\markboth{References}{References}

\phantomsection\label{refs}
\begin{CSLReferences}{1}{0}
\bibitem[\citeproctext]{ref-clark1993}
Clark, W. G. 1993. {``The Effect of Recruitment Variability on the
Choice of a Target Level of Spawning Biomass Per Recruit.''} In, 233246.
Alaska Sea Grant College Program
AK{\textendash}SG{\textendash}93{\textendash}02.

\end{CSLReferences}


\backmatter

\end{document}
