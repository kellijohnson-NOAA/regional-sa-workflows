% Options for packages loaded elsewhere
\PassOptionsToPackage{unicode}{hyperref}
\PassOptionsToPackage{hyphens}{url}
%
\documentclass[
  letterpaper,
  oneside,
  open=any]{scrbook}

\usepackage{amsmath,amssymb}
\usepackage{iftex}
\ifPDFTeX
  \usepackage[T1]{fontenc}
  \usepackage[utf8]{inputenc}
  \usepackage{textcomp} % provide euro and other symbols
\else % if luatex or xetex
  \usepackage{unicode-math}
  \defaultfontfeatures{Scale=MatchLowercase}
  \defaultfontfeatures[\rmfamily]{Ligatures=TeX,Scale=1}
\fi
\usepackage{lmodern}
\ifPDFTeX\else  
    % xetex/luatex font selection
\fi
% Use upquote if available, for straight quotes in verbatim environments
\IfFileExists{upquote.sty}{\usepackage{upquote}}{}
\IfFileExists{microtype.sty}{% use microtype if available
  \usepackage[]{microtype}
  \UseMicrotypeSet[protrusion]{basicmath} % disable protrusion for tt fonts
}{}
\makeatletter
\@ifundefined{KOMAClassName}{% if non-KOMA class
  \IfFileExists{parskip.sty}{%
    \usepackage{parskip}
  }{% else
    \setlength{\parindent}{0pt}
    \setlength{\parskip}{6pt plus 2pt minus 1pt}}
}{% if KOMA class
  \KOMAoptions{parskip=half}}
\makeatother
\usepackage{xcolor}
\setlength{\emergencystretch}{3em} % prevent overfull lines
\setcounter{secnumdepth}{5}
% Make \paragraph and \subparagraph free-standing
\ifx\paragraph\undefined\else
  \let\oldparagraph\paragraph
  \renewcommand{\paragraph}[1]{\oldparagraph{#1}\mbox{}}
\fi
\ifx\subparagraph\undefined\else
  \let\oldsubparagraph\subparagraph
  \renewcommand{\subparagraph}[1]{\oldsubparagraph{#1}\mbox{}}
\fi

\providecommand{\tightlist}{%
  \setlength{\itemsep}{0pt}\setlength{\parskip}{0pt}}\usepackage{longtable,booktabs,array}
\usepackage{calc} % for calculating minipage widths
% Correct order of tables after \paragraph or \subparagraph
\usepackage{etoolbox}
\makeatletter
\patchcmd\longtable{\par}{\if@noskipsec\mbox{}\fi\par}{}{}
\makeatother
% Allow footnotes in longtable head/foot
\IfFileExists{footnotehyper.sty}{\usepackage{footnotehyper}}{\usepackage{footnote}}
\makesavenoteenv{longtable}
\usepackage{graphicx}
\makeatletter
\def\maxwidth{\ifdim\Gin@nat@width>\linewidth\linewidth\else\Gin@nat@width\fi}
\def\maxheight{\ifdim\Gin@nat@height>\textheight\textheight\else\Gin@nat@height\fi}
\makeatother
% Scale images if necessary, so that they will not overflow the page
% margins by default, and it is still possible to overwrite the defaults
% using explicit options in \includegraphics[width, height, ...]{}
\setkeys{Gin}{width=\maxwidth,height=\maxheight,keepaspectratio}
% Set default figure placement to htbp
\makeatletter
\def\fps@figure{htbp}
\makeatother
% definitions for citeproc citations
\NewDocumentCommand\citeproctext{}{}
\NewDocumentCommand\citeproc{mm}{%
  \begingroup\def\citeproctext{#2}\cite{#1}\endgroup}
\makeatletter
 % allow citations to break across lines
 \let\@cite@ofmt\@firstofone
 % avoid brackets around text for \cite:
 \def\@biblabel#1{}
 \def\@cite#1#2{{#1\if@tempswa , #2\fi}}
\makeatother
\newlength{\cslhangindent}
\setlength{\cslhangindent}{1.5em}
\newlength{\csllabelwidth}
\setlength{\csllabelwidth}{3em}
\newenvironment{CSLReferences}[2] % #1 hanging-indent, #2 entry-spacing
 {\begin{list}{}{%
  \setlength{\itemindent}{0pt}
  \setlength{\leftmargin}{0pt}
  \setlength{\parsep}{0pt}
  % turn on hanging indent if param 1 is 1
  \ifodd #1
   \setlength{\leftmargin}{\cslhangindent}
   \setlength{\itemindent}{-1\cslhangindent}
  \fi
  % set entry spacing
  \setlength{\itemsep}{#2\baselineskip}}}
 {\end{list}}
\usepackage{calc}
\newcommand{\CSLBlock}[1]{\hfill\break\parbox[t]{\linewidth}{\strut\ignorespaces#1\strut}}
\newcommand{\CSLLeftMargin}[1]{\parbox[t]{\csllabelwidth}{\strut#1\strut}}
\newcommand{\CSLRightInline}[1]{\parbox[t]{\linewidth - \csllabelwidth}{\strut#1\strut}}
\newcommand{\CSLIndent}[1]{\hspace{\cslhangindent}#1}

\usepackage[default]{opensans}
\fontseries{lc}\selectfont
\definecolor{nmfsblue1}{HTML}{00467f}
\definecolor{nmfsblue2}{HTML}{007eb2}
\makeatletter
\@ifpackageloaded{bookmark}{}{\usepackage{bookmark}}
\makeatother
\makeatletter
\@ifpackageloaded{caption}{}{\usepackage{caption}}
\AtBeginDocument{%
\ifdefined\contentsname
  \renewcommand*\contentsname{Table of contents}
\else
  \newcommand\contentsname{Table of contents}
\fi
\ifdefined\listfigurename
  \renewcommand*\listfigurename{List of Figures}
\else
  \newcommand\listfigurename{List of Figures}
\fi
\ifdefined\listtablename
  \renewcommand*\listtablename{List of Tables}
\else
  \newcommand\listtablename{List of Tables}
\fi
\ifdefined\figurename
  \renewcommand*\figurename{Figure}
\else
  \newcommand\figurename{Figure}
\fi
\ifdefined\tablename
  \renewcommand*\tablename{Table}
\else
  \newcommand\tablename{Table}
\fi
}
\@ifpackageloaded{float}{}{\usepackage{float}}
\floatstyle{ruled}
\@ifundefined{c@chapter}{\newfloat{codelisting}{h}{lop}}{\newfloat{codelisting}{h}{lop}[chapter]}
\floatname{codelisting}{Listing}
\newcommand*\listoflistings{\listof{codelisting}{List of Listings}}
\makeatother
\makeatletter
\makeatother
\makeatletter
\@ifpackageloaded{caption}{}{\usepackage{caption}}
\@ifpackageloaded{subcaption}{}{\usepackage{subcaption}}
\makeatother

\usepackage{hyphenat}
\usepackage{ifthen}
\usepackage{calc}
\usepackage{calculator}

\usepackage{graphicx}
\usepackage{wallpaper}

\usepackage{geometry}

\usepackage{graphicx}
\usepackage{geometry}
\usepackage{afterpage}
\usepackage{tikz}
\usetikzlibrary{calc}
\usetikzlibrary{fadings}
\usepackage[pagecolor=none]{pagecolor}


% Set the titlepage font families







% Set the coverpage font families

\ifLuaTeX
  \usepackage{selnolig}  % disable illegal ligatures
\fi
\usepackage{bookmark}

\IfFileExists{xurl.sty}{\usepackage{xurl}}{} % add URL line breaks if available
\urlstyle{same} % disable monospaced font for URLs
\hypersetup{
  pdftitle={Summary of U.S. Stock Assessment Workflows},
  pdfauthor={Samantha Schiano},
  hidelinks,
  pdfcreator={LaTeX via pandoc}}

\title{Summary of U.S. Stock Assessment Workflows}
\usepackage{etoolbox}
\makeatletter
\providecommand{\subtitle}[1]{% add subtitle to \maketitle
  \apptocmd{\@title}{\par {\large #1 \par}}{}{}
}
\makeatother
\subtitle{Tools, Templates, and other Resources}
\author{Samantha Schiano}
\date{2024-04-01}

\begin{document}
%%%%% begin titlepage extension code

  \begin{frontmatter}

\begin{titlepage}
% This is a combination of Pandoc templating and LaTeX
% Pandoc templating https://pandoc.org/MANUAL.html#templates
% See the README for help

\thispagestyle{empty}

\newgeometry{top=-100in}

% Page color

\newcommand{\coverauthorstyle}[1]{{\fontsize{20}{24.0}\selectfont
#1}}

\begin{tikzpicture}[remember picture, overlay, inner sep=0pt, outer sep=0pt]

\tikzfading[name=fadeout, inner color=transparent!0,outer color=transparent!100]
\tikzfading[name=fadein, inner color=transparent!100,outer color=transparent!0]
\node[anchor=south west, rotate=0.0, opacity=1.0] at ($(current page.south west)+(0pt, 8.75in)$) {
\includegraphics[width=\paperwidth, keepaspectratio]{images/cover-header-2.png}};

% Title
\newcommand{\titlelocationleft}{2.3in}
\newcommand{\titlelocationbottom}{7in}
\newcommand{\titlealign}{left}

\begin{scope}{%
\fontsize{50}{60.0}\selectfont
\node[anchor=north
west, align=left, rotate=0] (Title1) at ($(current page.south west)+(\titlelocationleft,\titlelocationbottom)$)  [text width = 5in]  {\textcolor{nmfsblue1}{\nohyphens{Summary
of U.S. Stock Assessment Workflows}}};
}
\end{scope}

% Author
\newcommand{\authorlocationleft}{2.3in}
\newcommand{\authorlocationbottom}{5in}
\newcommand{\authoralign}{left}

\begin{scope}
{%
\fontsize{20}{24.0}\selectfont
\node[anchor=north
west, align=left, rotate=0] (Author1) at ($(current page.south west)+(\authorlocationleft,\authorlocationbottom)$)  [text width = 5in]  {
\coverauthorstyle{Samantha Schiano\\}};
}
\end{scope}

% Header
\newcommand{\headerlocationleft}{2.3in}
\newcommand{\headerlocationbottom}{9.8in}
\newcommand{\headerlocationalign}{left}

\begin{scope}
{%
\fontsize{16}{19.2}\selectfont
 \node[anchor=north west, align=left, rotate=0] (Header1) at %
($(current page.south west)+(\headerlocationleft,\headerlocationbottom)$)  [text width = 5in]  {\textcolor{white}{\nohyphens{NOAA
Technical Memorandum NMFS-XXX-\#\#}}};
}
\end{scope}

% Footer
\newcommand{\footerlocationleft}{6in}
\newcommand{\footerlocationbottom}{0.1\paperheight}
\newcommand{\footerlocationalign}{left}

\begin{scope}
{%
\fontsize{8}{9.6}\selectfont
 \node[anchor=north west, align=left, rotate=0] (Footer1) at %
($(current page.south west)+(\footerlocationleft,\footerlocationbottom)$)  [text width = 2.5in]  {{\nohyphens{U.S.
DEPARTMENT OF COMMERCE\\
\strut \\
National Oceanic and Atmospheric Administration\\
National Marine Fisheries Service\\
Northwest Fisheries Science Center}}};
}
\end{scope}

% Date
\newcommand{\datelocationleft}{6in}
\newcommand{\datelocationbottom}{2in}
\newcommand{\datelocationalign}{left}

\begin{scope}
{%
\fontsize{20}{24.0}\selectfont
 \node[anchor=north west, align=left, rotate=0] (Date1) at %
($(current page.south west)+(\datelocationleft,\datelocationbottom)$)  [text width = 2.5in]  {{\nohyphens{January
2023}}};
}
\end{scope}

\end{tikzpicture}
\clearpage
\restoregeometry
%%% TITLE PAGE START

% Set up alignment commands
%Page
\newcommand{\titlepagepagealign}{
\ifthenelse{\equal{left}{right}}{\raggedleft}{}
\ifthenelse{\equal{left}{center}}{\centering}{}
\ifthenelse{\equal{left}{left}}{\raggedright}{}
}
%% Titles
\newcommand{\titlepagetitlealign}{
\ifthenelse{\equal{left}{right}}{\raggedleft}{}
\ifthenelse{\equal{left}{center}}{\centering}{}
\ifthenelse{\equal{left}{left}}{\raggedright}{}
\ifthenelse{\equal{left}{spread}}{\makebox[\linewidth][s]}{}
}


\newcommand{\titleandsubtitle}{
% Title and subtitle
{\fontsize{30}{36.0}\selectfont
\textcolor{nmfsblue1}{\bfseries{\nohyphens{Summary of U.S. Stock
Assessment Workflows}}}\par
}%

\vspace{\betweentitlesubtitle}
{
{\textit{\nohyphens{Tools, Templates, and other Resources}}}\par
}}
\newcommand{\titlepagetitleblock}{
\titleandsubtitle
}

\newcommand{\authorstyle}[1]{{\fontsize{20}{24.0}\selectfont
#1}}

\newcommand{\affiliationstyle}[1]{{#1}}

\newcommand{\titlepageauthorblock}{
{\authorstyle{\nohyphens{Samantha Schiano}{\textsuperscript{1}}}}}

\newcommand{\titlepageaffiliationblock}{
\hangindent=1em
\hangafter=1
{\affiliationstyle{
{1}.~ECS Federal in Support of,~NOAA Fisheries Office of Science and
Technology,~1315 East West HighwaySilver Spring, MD 20910


\vspace{1\baselineskip} 
}}
}
\newcommand{\headerstyled}{%
{}
}
\newcommand{\footerstyled}{%
{}
}
\newcommand{\datestyled}{%
{2024-04-01}
}


\newcommand{\titlepageheaderblock}{\headerstyled}

\newcommand{\titlepagefooterblock}{
\footerstyled
}

\newcommand{\titlepagedateblock}{
\datestyled
}

%set up blocks so user can specify order
\newcommand{\titleblock}{\newlength{\betweentitlesubtitle}
\setlength{\betweentitlesubtitle}{1pt}
{\titlepagetitlealign

{\titlepagetitleblock}
}

\vspace{4\baselineskip}
}

\newcommand{\authorblock}{{\titlepageauthorblock}

\vspace{2\baselineskip}
}

\newcommand{\affiliationblock}{{\titlepageaffiliationblock}

\vspace{2\baselineskip}
}

\newcommand{\logoblock}{}

\newcommand{\footerblock}{}

\newcommand{\dateblock}{{\titlepagedateblock}

\vspace{0pt}
}

\newcommand{\headerblock}{}
\newgeometry{top=3in,bottom=1in,right=1in,left=1.75in}
% background image
\newlength{\bgimagesize}
\setlength{\bgimagesize}{0.75\paperwidth}
\LENGTHDIVIDE{\bgimagesize}{\paperwidth}{\theRatio} % from calculator pkg
\ThisULCornerWallPaper{\theRatio}{images/corner-image.png}

\thispagestyle{empty} % no page numbers on titlepages


\newcommand{\vrulecode}{\rule{\vrulewidth}{\textheight}}
\newlength{\vrulewidth}
\setlength{\vrulewidth}{0pt}
\newlength{\B}
\setlength{\B}{\ifdim\vrulewidth > 0pt 0.05\textwidth\else 0pt\fi}
\newlength{\minipagewidth}
\ifthenelse{\equal{left}{left} \OR \equal{left}{right} }
{% True case
\setlength{\minipagewidth}{\textwidth - \vrulewidth - \B - 0.1\textwidth}
}{
\setlength{\minipagewidth}{\textwidth - 2\vrulewidth - 2\B - 0.1\textwidth}
}
\ifthenelse{\equal{left}{left} \OR \equal{left}{leftright}}
{% True case
\raggedleft % needed for the minipage to work
\vrulecode
\hspace{\B}
}{%
\raggedright % else it is right only and width is not 0
}
% [position of box][box height][inner position]{width}
% [s] means stretch out vertically; assuming there is a vfill
\begin{minipage}[b][\textheight][s]{\minipagewidth}
\titlepagepagealign
\headerblock

\titleblock

\authorblock

\affiliationblock

\vfill

\logoblock

\footerblock
\par

\end{minipage}\ifthenelse{\equal{left}{right} \OR \equal{left}{leftright} }{
\hspace{\B}
\vrulecode}{}
\clearpage
\restoregeometry
%%% TITLE PAGE END
\end{titlepage}
\setcounter{page}{1}
\end{frontmatter}

%%%%% end titlepage extension code

\renewcommand*\contentsname{Table of contents}
{
\setcounter{tocdepth}{1}
\tableofcontents
}
\listoffigures
\listoftables
\mainmatter
\bookmarksetup{startatroot}

\chapter*{Citation}\label{citation}
\addcontentsline{toc}{chapter}{Citation}

\markboth{Citation}{Citation}

Schiano, S. 2024. Summary of U.S. Stock Assessment Workflows: Tools and
Templates. NOAA Fisheries Office of Science and Technology.

\bookmarksetup{startatroot}

\chapter{Summary}\label{summary}

Developing and producing a stock assessment report requires a
considerable amount of data consolidation, analysis, and research.
Reports can range from 10 pages to well over 300, but the goal of the
process remains to provide management advice with the current and
projected status of the stock, catches, and other important parameters
to ensure its sustainability. There have been many efforts done to
improve the reproducibility of workflows and reduce time it takes to
produce these reports. So far, improvements and efforts have ranged
considerably by region of the U.S. Specifically, each of the seven
regional fishery science centers across the U.S. have their own
workflows to assess a stock and produce its report. Many of these
workflows are guided by requirements from fishery management councils
and other involved managing bodies that utilize these reports to
delegate fishery regulations.

Workflows across the country not only range from region to region, but
from scientist to scientist. There is no current standardized or
accepted best practice for producing a stock assessment report. Some
centers rely on the use of latex, a common software to produce documents
with in-text calculations, while others utilize more recently developed
programs like Rmarkdown and quarto, which both use latex as a basis for
their production. Others utilize the capacity of Microsoft word which
restricts the reproducibility of a report as well as increases work time
since figures, tables, and other associated resources must be compiled
outside of the word document.

\textbf{Description of interaction and list of fishery management
councils across the US}

\bookmarksetup{startatroot}

\chapter{General Stock Assessment Workflow (commonalities and
process)}\label{sec-general}

\begin{itemize}
\tightlist
\item
  Description of stock assessment workflows from input to bringing the
  report to the SSC and council(s) for evaluation and adoption of formal
  management measures resulting from recommendations
\end{itemize}

\begin{enumerate}
\def\labelenumi{\arabic{enumi}.}
\item
  Gather data inputs for model
\item
  Configure assessment model with updated data (based on assessment need
  for that year)
\item
  Sensitivity runs and projections
\item
  Develop assessment report for SSC, councils, and/or RFMOs
\item
  Present assessment and recommendations to SSC and councils
\item
  Assessment accepted or not, create formal report for public release
  and adoption recommendations as designated by the councils
\end{enumerate}

Identifiable issues:

\begin{itemize}
\item
  Inconsistent naming conventions for parameters
\item
  Inconsistent format/no guidelines present for a U.S. assessment report
\end{itemize}

\bookmarksetup{startatroot}

\chapter{Stock Assessment Models}\label{stock-assessment-models}

{[}Short descriptions of commonly used stock assessment models within
the U.S. including acknowledgement of smaller used models and FIMS for
the future{]}

\begin{itemize}
\item
  General use of assessment models in the workflow (self-explanatory)
\item
  Short descriptions of various models used in the U.S. around the
  regions (purpose is so that the reader can understand the model when
  it is reference in the section later)

  \begin{itemize}
  \tightlist
  \item
    Categorize based on assessment type rather than assessment model
    (age-structured/catch-at-age, catch-at-length, VPA, Agg. Biomass
    Dynamics, Index-based, data-limited)
  \end{itemize}
\item
  Include link/reference to papers/repositories at end of summary for
  reader to reference (also refer to FIT)
\item
  WHAM, SS, BAM, ASAP, AMAK, Bespoke, FIMS, ect
\end{itemize}

\bookmarksetup{startatroot}

\chapter{Tools and Resources}\label{sec-tools}

\begin{itemize}
\tightlist
\item
  Tools and templates available for different models and workflows
  (added as a list of tools in a way instead of by region since there
  are a lot of regions that use the same tools)
\end{itemize}

\section{afscdata}\label{sec-afscdata}

\section{afscassess}\label{sec-afscassess}

\section{safe Report Template}\label{sec-safe}

\section{r4ss}\label{sec-r4ss}

\section{sa4ss - SS Report Template}\label{sec-sa4ss}

\section{pfmc-assessments}\label{sec-pfmcassess}

\section{ASAPplots}\label{sec-asapplots}

\section{MAFSC SAFE Reports Template}\label{sec-mafscsafe}

\section{SEDAR-Assessement Report Template}\label{sec-sedartemplate}

\section{FishGraph}\label{sec-fishgraph}

\section{ADMB2R}\label{sec-admb2r}

\section{SASINF}\label{sec-sasinf}

\section{SW Stock Assessment Template}\label{sec-swfsctemplate}

\section{BrailleR}\label{sec-brailler}

\section{SW R Process Output}\label{sec-swroutput}

\section{swfscMisc}\label{sec-swfscmisc}

\section{NMFSReports}\label{sec-nmfsreports}

\section{NOAA Tech Memo Template}\label{sec-noaatechmemo}

\section{Other Assessment Report Template
Repositories}\label{sec-othertemplates}

\bookmarksetup{startatroot}

\chapter{Region Specific Workflow}\label{sec-regionworkflows}

\begin{itemize}
\item
  Descriptions of the workflows by region and what separates them from
  other regions
\item
  Advances made by this region and the tools they use
\item
  Particular struggles or unique operations incorporated into their
  workflow
\item
  Table of resources used for their workflows
\end{itemize}

Note: assessment reports and incorporated materials are all assessment
author based outside of their regional requirements

Alternative way to reference section \hyperref[sec-safe]{safe Report
Template} rather than Section~\ref{sec-safe} *Note doesn't work with
html render.

\section{AFSC}\label{sec-afsc}

Process variations:

\begin{itemize}
\item
  Uses afscData (Section~\ref{sec-afscdata}) for data extractions for
  model inputs

  \begin{itemize}
  \tightlist
  \item
    SQL for data query
  \end{itemize}
\item
  AFSCassess (Section~\ref{sec-afscassess}) for cleaning up data,
  generating figures, and other associated processes with generating a
  stock assessment report
\item
  Functions tailored to pull data/perform function for each species
\item
  ADMB model outputs
\item
  Utilizes `safe' (Section~\ref{sec-safe}) reporting template

  \begin{itemize}
  \tightlist
  \item
    Automated and reproducible, large effort put into making and
    maintaining
  \end{itemize}
\item
  Formalized guidelines for reports
\end{itemize}

Largest problems:

\begin{itemize}
\item
  Inheriting an assessment can be a challenge
\item
  Data structures very different in GOA v. BSAI
\item
  Input and output framework needs work
\item
  Large amount of tables and figures (potential to clog the process; is
  this too many?)
\end{itemize}

\section{NEFSC}\label{sec-nefsc}

Process variations:

\begin{itemize}
\item
  Informal/verbal agreement for TOR guidelines for each stock
\item
  Standardized report template agreed upon by NOAA and the MAFMC AND
  NEFMC (agreed on in 2017)

  \begin{itemize}
  \tightlist
  \item
    Short and concise to make policy decisions (mgmt track specifically)
  \end{itemize}
\item
  Report template is not publicly available but all done in latex
  (modular workflow)

  \begin{itemize}
  \item
    Figures rendered outside (saved as png) then reference in doc
  \item
    Tables created into tex files and referenced as component in
    template
  \end{itemize}
\item
  Extensive work with 508 compliance

  \begin{itemize}
  \tightlist
  \item
    Contractors developing package for compliance to apply to template
  \end{itemize}
\end{itemize}

Largest problems:

\begin{itemize}
\item
  Lots of processing variables for 508 compliance
\item
  Even with all the work in the compliance, there is still a large
  effort into making it accessible (\textasciitilde2 week conversion for
  single analyst at the center)
\item
  Workflow still labor intensive
\item
  Large barrier to use template for new users
\item
  Only automated for management track reports
\end{itemize}

\section{NWFSC}\label{sec-nwfsc}

Process Variations:

\begin{itemize}
\item
  Input data extractions are from an online database and state agencies

  \begin{itemize}
  \tightlist
  \item
    nwfscSurvey repo for west coast groundfish survey data
  \end{itemize}
\item
  Use of sa4ss (maintained by scientist at NWFSC), package to make a
  template in github
\item
  All figures generated using r4ss functions
\item
  Standardized tables
\item
  Scripts are specific to species
\item
  Use of SS in the NW led to huge developments for packages that are
  tuned to its output such as r4ss and sa4ss
\item
  Process is not standardized for entire center (common among a lot of
  centers)

  \begin{itemize}
  \tightlist
  \item
    Some assessment scientists create a new repo for each assessment
  \end{itemize}
\end{itemize}

Largest Issues:

\begin{itemize}
\item
  Reproducibility
\item
  Iterative process of incorporating changes to models for report
  generation: not optomized for this part of the workflow
\end{itemize}

\section{PIFSC}\label{sec-pifsc}

Process variations:

\begin{itemize}
\item
  ``Template'' report in github (private and not modular; international)
\item
  No generalized workflow/process (aka each species is a little
  different)
\item
  Workflow isn't specifically defined
\item
  Process tables and figures in R then incorporate into a word document
  (domestic)

  \begin{itemize}
  \tightlist
  \item
    Reported out as a tech memo
  \end{itemize}
\end{itemize}

Largest Issues:

\begin{itemize}
\item
  Work with international and species complexes that make it difficult
  to have a general reporting structure
\item
  Barrier to entry with quarto
\item
  Can't update .qmd file after changes made to word doc after render
\item
  Parallelization of workflows
\end{itemize}

\section{SEFSC}\label{sec-sefsc}

Process Variations:

\begin{itemize}
\item
  Follows requirements for SEDAR
\item
  Figures and tables at the end of the report (SEDAR requirement)
\item
  File system on server where data is stored
\end{itemize}

Caribbean:

\begin{itemize}
\item
  A lot of intermediaries in the process (aka figures and tables
  generated outside of the report)
\item
  Incorporated some use of Rmarkdown with a modularized template system
\item
  Prototype template using quarto books for SEDAR 57
\end{itemize}

Gulf of Mexico:

\begin{itemize}
\item
  Rmarkdown template used in process (private repository on GitHub)
\item
  Work with SS report file and automate from there
\item
  Some parameters are hard coded and changed per species
\item
  New forked repo. for each stock assessment
\end{itemize}

South Atlantic:

\begin{itemize}
\item
  Data is gathered as an excel file - manually converted as inputs for
  the assessment model (potential for automation but analysts lack time)
\item
  Model process includes: fine tuning, diagnostics, ADMB2R
  (Section~\ref{sec-admb2r}), FishGraph (Section~\ref{sec-fishgraph})
\item
  Uses latex template with tables as outside tex doc (incorporated into
  the template) and figures produced outside of template into a folder
  then referenced into template
\item
  Report not completely automated
\item
  Organized on a species by species case
\item
  Template not on github
\item
  Previous reports are shared with analyst who next inherits it through
  backup drive
\end{itemize}

Largest Issues:

\begin{itemize}
\item
  Variable answer based on region (South Atlantic v. Gulf of Mexico v.
  Caribbean)
\item
  Confidentiality (large recreational base)
\item
  Data providers don't provide data in the same format
\item
  Limited to no 508 compliance (other than minimum requirements)

  \begin{itemize}
  \tightlist
  \item
    Highly time intensive
  \end{itemize}
\end{itemize}

Caribbean:

\begin{itemize}
\item
  Difficulty referencing tables and figures in Rmarkdown
\item
  Limited abilities and knowledge with R/quarto
\item
  Oftentimes updates seem to feel like benchmark assessments due to
  requests
\end{itemize}

Gulf of Mexico:

\begin{itemize}
\item
  Figures in regards to inputs cause most issues
\item
  Long, large reports
\item
  Steep learning curve for those not yet using markdown (all in the
  branch have at least used the Rmarkdown template once for an
  assessment)
\item
  Large data sets/data takes a while to read in
\item
  Cannot access others' SS output files dues to confidentiality (those
  outside of NOAA org)
\item
  Projections aren't standardized
\end{itemize}

South Atlantic:

\begin{itemize}
\item
  Process is not automated
\item
  Localized reporting template
\end{itemize}

\section{SWFSC}\label{sec-swfsc}

Process Variations:

\begin{itemize}
\item
  Inherited workflow from analysts
\item
  Use of `sa4ss' (Section~\ref{sec-sa4ss}) for workflow with some
  modifications
\item
  Centralized Rmarkdown for process
\item
  Access to past assessments on PFMC GitHub
\end{itemize}

Largest Issues:

\begin{itemize}
\item
  Data acquisition and handling (also troubles associated with data
  confidentiality)
\item
  Tough transition for new employees due GitHub used on an
  analyst-to-analyst bases and unclear workflow structure
\item
  Model versioning and changes to the report through a non-linear
  process (common among other regions too)
\item
  508 compliance efforts sometimes break
\end{itemize}

Example in text reference (Clark 1993).

\bookmarksetup{startatroot}

\chapter{Conclusions}\label{conclusions}

\begin{itemize}
\item
  Conclusions about how workflows operate in the U.S.

  \begin{itemize}
  \item
    Both pros and pitfalls of current workflows
  \item
    Incorporating perspectives from regional scientists?
  \end{itemize}
\item
  Additional hope for greater look into the inputs and pulling together
  the entire stock assessment workflow process
\end{itemize}

\section{Future Work}\label{future-work}

\begin{itemize}
\tightlist
\item
  Discuss the upcoming/in development tool for automated workflows and
  plan for the future including stock assessment modelling for the
  future
\end{itemize}

\bookmarksetup{startatroot}

\chapter*{References}\label{references}
\addcontentsline{toc}{chapter}{References}

\markboth{References}{References}

\phantomsection\label{refs}
\begin{CSLReferences}{1}{0}
\bibitem[\citeproctext]{ref-clark1993}
Clark, W. G. 1993. {``The Effect of Recruitment Variability on the
Choice of a Target Level of Spawning Biomass Per Recruit.''} In, 233246.
Alaska Sea Grant College Program
AK{\textendash}SG{\textendash}93{\textendash}02.

\end{CSLReferences}


\backmatter

\end{document}
